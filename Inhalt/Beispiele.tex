\section{Folien}

  \subsection{Folie mit Titel}
  \begin{frame}[fragile]{\subsecname}

    \begin{lstlisting}[gobble=6,escapechar=\%]
      %\%% Folie mit Titel

      %\textbackslash%begin{frame}{Folie mit Titel}
      %\textbackslash%end{frame}
    \end{lstlisting}

  \end{frame}

  \subsection{Folie ohne Titel}
  \begin{frame}[fragile]

    \begin{lstlisting}[gobble=6,escapechar=\%]
      %\%% Folie ohne Titel
      
      %\textbackslash%begin{frame}
      %\textbackslash%end{frame}
    \end{lstlisting}
    
  \end{frame}

  \subsection{Folie mit Option [plain]}
  \begin{frame}[fragile,plain]

    \begin{lstlisting}[gobble=6,escapechar=\%]
      %\%% Folie mit Option [plain]

      %\textbackslash%begin[plain]{frame}
      %\textbackslash%end{frame}
    \end{lstlisting}

  \end{frame}

\section{Text \& Auflistungen}

  \subsection{Lorem}
  \begin{frame}{\subsecname}
    \blindtext
  \end{frame}

  \subsection{Ipsum}
  \begin{frame}{\subsecname}
    \blinditemize
  \end{frame}

  \subsection{Text}
  \begin{frame}{\subsecname}
    \textbf{textbf}
    \emph{emph}
    \structure{structure}
    \alert{alert}
    \color{blue}{blue}
  \end{frame}
  
  \subsection{Auflistungen}
  \begin{frame}{\subsecname}

    \begin{minipage}[t]{0.5\textwidth}
      \structure{Enumerate}
      \par\medskip
      \begin{enumerate}
        \item eins
        \item zwei
        \item drei
      \end{enumerate}
    \end{minipage}%
    \begin{minipage}[t]{0.5\textwidth}
      \structure{Itemize}
      \par\medskip
      \begin{itemize}
        \item eins
        \item zwei
        \item drei
      \end{itemize}
    \end{minipage}

  \end{frame}
  
\section{Formeln}

  \subsection{In-Line}
  \begin{frame}{\subsecname}

    Beispiel: \( E = mc^2 \)
  
  \end{frame}
  
  \subsection{Extra Line}
  \begin{frame}{\subsecname}

    Beispiel:

    \begin{equation*}
      x_{1,2} = -\frac{p}{2} \pm \sqrt{ \left( \frac{p}{2} \right)^2 - q }
    \end{equation*}

  \end{frame}

\section{Listings}

  \subsection{In-Line}
  \begin{frame}[fragile]{\subsecname}

    \begin{lstlisting}[gobble=6]
      % Option [fragile] ist erforderlich!

      \lstinline! ... !
    \end{lstlisting}

  \end{frame}

  \subsection{Extra Line}
  \begin{frame}[fragile]{\subsecname}

    \begin{lstlisting}[gobble=6,escapechar=\%]
      %\%% Option [fragile] ist erforderlich!

      %\textbackslash%begin{lstlisting}
      %\textbackslash%end{lstlisting}
    \end{lstlisting}

  \end{frame}

  \subsection{Extra Frame}
  \begin{frame}[fragile]{\subsecname}

    \begin{lstlisting}[gobble=6]
      % Option [fragile] ist erforderlich!

      \lstinputlisting{Verzeichnis/Dateiname}
    \end{lstlisting}

  \end{frame}

\section{Spalten \& Blöcke}

  \subsection{Spalten}
  \begin{frame}{\subsecname}

    \begin{minipage}[t]{0.5\textwidth}
      \structure{Links}
      \par\medskip
      Inhalt der linken Spalte.
    \end{minipage}%
    \begin{minipage}[t]{0.5\textwidth}
      \structure{Rechts}
      \par\medskip
      Inhalt der rechten Spalte.
    \end{minipage}

  \end{frame}

  \subsection{Blöcke}
  \begin{frame}[fragile]{\subsecname}

    \begin{block}{Block}
      \begin{lstlisting}[gobble=8]
        \begin{block}{Titel}
          Text
        \end{block}
      \end{lstlisting}
    \end{block}

    \begin{block}{Block}
      \begin{lstlisting}[gobble=8]
        \begin{block}{Titel}
          Text
        \end{block}
      \end{lstlisting}
    \end{block}

  \end{frame}

  \subsection{Blöcke}
  \begin{frame}[fragile]{\subsecname}

    \begin{exampleblock}{Exampleblock}
      \begin{lstlisting}[gobble=8]
        \begin{exampleblock}{Titel}
          Text
        \end{exampleblock}
      \end{lstlisting}
    \end{exampleblock}

    \begin{alertblock}{Alertblock}
      \begin{lstlisting}[gobble=8]
        \begin{alertblock}{Titel}
          Text
        \end{alertblock}
      \end{lstlisting}
    \end{alertblock}

  \end{frame}